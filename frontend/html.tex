\documentclass[12pt,a4paper]{beamer}
\usepackage[utf8]{inputenc}
\usepackage{graphicx}
\usepackage{listings}
\usepackage[spanish]{babel}
\usepackage[T1]{fontenc}
\usepackage{parskip}
\usepackage{framed}

\author{Mario Lamas}
\title{HTML 5}
\date{}

\begin{document}

\begin{frame}
	\maketitle
\end{frame}

\begin{frame}
	\frametitle{Introducción}
	\lstinputlisting[language=HTML]{html/01.html}
\end{frame}

\begin{frame}
	\frametitle{¿Qué es HTML?}
	
	HTML es un lenguage para describir páginas web.
	\begin{itemize}
		\item HTML significa Hyper Text Markup Language \pause
		\item HTML es un lenguage de marcado \pause
		\item Un lenguage de marcado es un conjunto de etiquetas de marcado \pause
		\item Las etiquetas describen el contenido del documento \pause
		\item Los documentos HTML pueden contener etiquetas HTML y texto plano \pause
		\item Los documentos HTML son llamados también \textbf{páginas web}
	\end{itemize}
\end{frame}

\begin{frame}
	\frametitle{Etiquetas HTML}
	\begin{itemize}
		\item Las etiquetas HTML son palabras clave (el nombre de las etiquetas) rodeada por \textbf{signos de mayor y menor} como <html> \pause
		\item Las etiquetas HTML \textbf{generalmente vienen en pares} como <b> y </b> \pause
		\item La primera etiqueta en el par es la \textbf{etiqueta de inicio} y la segunda etiqueta es la \textbf{etiqueta final} \pause
		\item La etiqueta final es escrita como la etiqueta de inicio con un slash antes del nombre de la etiqueta \pause
		\item Las etiquetas inicial y final se les llama también \textbf{etiqueta de apertura} y \textbf{etiqueta de clausura} \pause
		\item Ejemplo: <nombre>contenido</nombre>
	\end{itemize}	
\end{frame}

\begin{frame}
	\frametitle{Versiones HTML}
	
	Desde los inicios de web se han tenido muchas versiones de HTML
	
	\begin{center}
		\begin{tabular}{|l|l|}
			\hline
			Versión & 	Año\\
			\hline
			HTML	&	1991\\
			HTML+	&	1993\\
			HTML 2.0	&	1995\\
			HTML 3.2	&	1997\\
			HTML 4.01	&	1999\\
			XHTML	&	2000\\
			HTML 5	&	2012\\
			\hline
		\end{tabular}
	\end{center}
	
\end{frame}

\begin{frame}
	\frametitle{La declaración <!DOCTYPE>}
	
	La declaración <!DOCTYPE> ayuda al navegador a mostrar la página correctamente.	
\end{frame}

\begin{frame}
	\frametitle{Declaraciones comunes}
	
	HTML5
	\begin{framed}
		<!DOCTYPE html>
	\end{framed}

	
	HTML 4.01
	\begin{framed}
		<!DOCTYPE HTML PUBLIC "-//W3C//DTD HTML 4.01 Transitional//EN" "http://www.w3.org/TR/html4/loose.dtd">
	\end{framed}	
	
	

\end{frame}

\begin{frame}
	\frametitle{Declaraciones comunes}
	
	XHTML 1.0
	\begin{framed}
		<!DOCTYPE html PUBLIC "-//W3C//DTD XHTML 1.0 Transitional//EN" "http://www.w3.org/TR/xhtml1/DTD/xhtml1-transitional.dtd"> 
	\end{framed}
\end{frame}

\begin{frame}
	\frametitle{Elementos HTML}
	
	\begin{itemize}
		\item Los documentos HTML están definidos por los elementos HTML \pause
		\item Un elemento HTML es todo desde la etiqueta de inicio hasta la etiqueta final		
	\end{itemize}
	
	\begin{center}
		\begin{tabular}{|p{4cm}|p{3.5cm}|p{2.5cm}|}
			\hline
			Etiqueta de inicio	&	Contenido del elemento	&	Etiqueta final\\
			\hline
			<p>	&	Esto es un párrafo	&	</p>\\
			<a href=``default.html''>	&	Esto es un enlace	&	</a>\\
			<br>	&	&\\
			\hline		
		\end{tabular}
	\end{center}	
\end{frame}

\begin{frame}
	\frametitle{Sintaxis del elemento HTML}
	
	\begin{itemize}
		\item Un elemento HTML empieza con una etiqueta de inicio o etiqueta de apertura \pause
		\item Un elemento HTML acaba con una etiqueta de finalización o etiqueta de cierre \pause
		\item El contenido del elemento es todo entre la etiqueta de inicio y la etiqueta final \pause
		\item Algunos elementos HTML tienen contenido vacío \pause
		\item Los elementos vacíos son cerrados en la etiqueta de inicio \pause
		\item La mayoría de los elementos HTML tienen atributos
	\end{itemize}
\end{frame}

\begin{frame}
	\frametitle{Elementos HTML anidados}
	
	La mayoría de los elementos HTML pueden ser anidados (pueden contener otro elementos HTML).
	
	El documento HTML consiste de elementos HTML anidados.
\end{frame}

\begin{frame}
	\frametitle{Ejemplo de documento HTML}
	
	\lstinputlisting{html/02.html}
	
	El ejemplo de arriba contiene 3 elementos HTML.
\end{frame}

\begin{frame}
	\frametitle{Explicación}
	
	El elemento <p>:
	
	\begin{framed}
		<p>This is my first paragraph</p>
	\end{framed}
	
	El elemento <p> define un párrafo en el documento HTML.
	
	El elemento <body>:
	
	\begin{framed}
		<body>\\
			<p>This is my first paragraph</p>\\
		</body>
	\end{framed}
	
	El elemento <body> define el cuerpo del documento HTML.
\end{frame}

\begin{frame}
	\frametitle{Explicación}
	
	El elemento <html>:
	
	\begin{framed}
		<html>\\
			<body>\\
				<p>This is my first paragraph.</p>\\
			</body>\\
		</html>
	\end{framed}
	
	El elemento <html> define todo el documento HTML.
\end{frame}

\begin{frame}
	\frametitle{No olvides la etiqueta de finalización}
	
	Algunos elemento HTML se pueden mostrar correctamente \textbf{aún si olvidas la etiqueta de finalización}.
	
	\begin{framed}
		<p>This is a paragraph\\
		<p>This is a paragraph
	\end{framed}
	
	El ejemplo de arriba trabaja en la mayoría de los navegadores, porque la etiqueta de finalización se considera opcional.
	
	Nunca hay que confiar en esto. Muchos elementos HTML producirán resultados inesperados y/o errores si olvidas la etiqueta de finalización.
\end{frame}

\begin{frame}
	\frametitle{Elementos HTML vacíos}
	
	Los elementos HTML sin contenido son llamados elementos vacíos.
	
	<br> es un elemento vacío sin una etiqueta de finalización (<br> define un salto de linea).
	
	\textbf{Nota}: En XHTML, todos los elementos deben ser cerrados. Añadiendo un slash dentro de la etiqueta de inicialización, como <br/>, es la manera correcta de cerrar elementos vacíos en XHTML (y XML). 
\end{frame}

\begin{frame}
	\frametitle{Usar las etiquetas con letras minúsculas}
	
	Las etiquetas HTML no son sensibles a las mayúsculas. <P> significa los mismo que <p>.
	
	La World Wide Web Consortium recomienda minúsculas.
\end{frame}

\begin{frame}
	\frametitle{Atributos HTML}
	
	\begin{itemize}
		\item Los elementos HTML pueden tener atributos \pause
		\item Los atributos proveen información adicional acerca del elemento \pause
		\item Los atributos siempre son especificados en la etiqueta de inicio \pause
		\item Los atributos vienen en pares nombre/valor como: \textbf{nombre=``valor''}
	\end{itemize}
\end{frame}

 \begin{frame}
	\frametitle{Ejemplo}
	
	Los enlaces HTML están definidos con la etiqueta <a>. La dirección del enlace es especificada en el \textbf{atributo href}.
	
	\begin{framed}
		<a href=``http://www.python.org''>Esto es un enlace</a>
	\end{framed}
 \end{frame}
 
 \begin{frame}
	\frametitle{Los valores de los atributos van entre comillas}
	
	Los valores de los atributos siempre deben ir entre comillas.
	
	Pueden ser comillas dobles o simples.
	
	Hay situaciones raras como cuando dentro del valor hay comillas. En ese caso se tendrá que utilizar las comillas simples
	
	\begin{framed}
		<input name='John ``ShotGun'' Nelson' />
	\end{framed}
 \end{frame}

\end{document}